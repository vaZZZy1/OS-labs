
\documentclass[pdf, unicode, 12pt, a4paper,oneside,fleqn]{article}
\usepackage{graphicx}
\graphicspath{{img/}}
\usepackage{log-style}
\begin{document}

\begin{titlepage}
    \begin{center}
        \bfseries

        {\Large Московский авиационный институт\\ (национальный исследовательский университет)}
        
        \vspace{48pt}
        
        {\large Факультет информационных технологий и прикладной математики}
        
        \vspace{36pt}
        
        {\large Кафедра вычислительной математики и~программирования}
        
        \vspace{48pt}
        
        Лабораторная работа \textnumero 3 по курсу \enquote{Операционные системы}

        \vspace{48pt}

        Управление потоками в ОС. Обеспечение синхронизации между потоками
    \end{center}
    
    \vspace{140pt}
    
    \begin{flushright}
    \begin{tabular}{rl}
    Студент: & В.\,А. Амурский \\
    Преподаватель: & Е. \,С. Миронов \\
    Группа: & М8О-201Б-21 \\
    Вариант: & 16 \\
    Дата: & \\
    Оценка: & \\
    Подпись: & \\
    \end{tabular}
    \end{flushright}
    
    \vfill
    
    \begin{center}
    \bfseries
    Москва, \the\year
    \end{center}
\end{titlepage}
    
\pagebreak

\section{Постановка задачи}

Составить программу на языке Си, обрабатывающую данные в многопоточном
режиме. При обработки использовать стандартные средства создания потоков
операционной системы (Windows/Unix). Ограничение потоков должно быть задано 
ключом запуска вашей программы.

Так же необходимо уметь продемонстрировать количество потоков, используемое 
вашей программой с помощью стандартных средств операционной системы.

В отчете привести исследование зависимости ускорения и эффективности 
алгоритма от входящих данных и количества потоков. Получившиеся результаты 
необходимо объяснить.

Задается радиус окружности. Необходимо с помощью метода Монте-Карло расчитать ее площадь

\section{Сведения о программе}

Программа написанна на Си в Unix подобной операционной системе на базе ядра Linux.
Для компиляции требуется ключ -lpthread, для запуска программы нелбходимо указать
в качестве аргумента количество потоков, которые максимально могут быть использованы.

Программа считывает со стандартного потока ввода радиус окружности и количество потоков, которые нужно создать.

Программа выводит в стандартный поток вычисленную площадь окружности методом Монте-Карло

\section{Общий метод и алгоритм решения}

Введем систему координат, центром которой будет центр окружности. Мысленно очерчим вокруг 
окружности квадрат со стороной 2r, так, чтобы окружность целиком находилась внутри квадрата.

Пусть изначально n = 0.
Программа $10^7$ раз выбирает случайную точку внутри квадрата. Если эта точка находится внутри окружности, то значение n увеливается на 1. Искомая площадь будет равна $\frac{n}{10^7}$

Если m - количество используемых потоков, в таком случае количество попыток выбора случайной точки уменьшается до  $\frac{10^7}{m}$
\section{Листинг программы}

{\large\textbf{main.cpp}}

\begin{lstlisting}[language=C++]
#include <iostream>
#include <pthread.h>
#include <vector>
#include "utils.h"
#include "lab3.h"
#include <chrono>


int main() {
    int threadCount;
    double r;
    std::cin >> r >> threadCount;

    int total = 0, success = 0, limit = 1e7;
    srandom(time(nullptr));

    std::vector<pthread_t> p(threadCount);
    std::vector<Args> a;

    for (int i = 0; i < threadCount; i++) {
        a.push_back({r, total, success, (limit + threadCount - 1) / threadCount});
    }

    auto start = std::chrono::high_resolution_clock::now();

    for (int i = 0; i < threadCount; i++) {
        pthread_create(&p[i], nullptr, &CalculateArea, &a[i]);
    }

    for (int i = 0; i < threadCount; i++) {
        pthread_join(p[i], nullptr);
    }

    auto end = std::chrono::high_resolution_clock::now();
    auto searchTime = std::chrono::duration_cast<std::chrono::milliseconds>(end - start).count();


    int resSuccess = 0;
    int resTotal = 0;
    for (int i = 0; i < threadCount; i++) {
        resSuccess += a[i].success;
        resTotal += a[i].total;
    }

    std::cout << resSuccess * 4 * r * r / (double) resTotal << " " << searchTime;
    return 0;
}
\end{lstlisting}

{\large\textbf{lab3.cpp}}

\begin{lstlisting}[language=C++]
#include "lab3.h"
#include "utils.h"


bool InCircle(double x, double y, double r) {
    if (x * x + y * y <= r * r + EPS) {
        return true;
    }
    return false;
}

void *CalculateArea(void *args) {
    auto *a = (struct Args *) args;
    for (int i = 0; i < a->limit; i++) {
        a->total++;
        a->success += InCircle(GetRandomNumber(a->r),
                               GetRandomNumber(a->r),
                               a->r);
    }
    return a;
}
\end{lstlisting}


{\large\textbf{utils.cpp}}

\begin{lstlisting}[language=C++]
#include "utils.h"


double GetRandomNumber(double max) {
    return -max + (double) (random() % (long) 1e5) / 1e5 * max * 2;
}
\end{lstlisting}

\section{Демонстрация работы программы}

\begin{verbatim}
vazy1@vazy1-legion:~/ClionProjects/OS-labs/tests$ cat lab3_test.cpp 
#include <gtest/gtest.h>

#include <lab3.h>
#include <utils.h>
#include <cmath>


TEST(ThirdLabTests, GetRandomNumberCorrectResults) {
    EXPECT_LE(abs(GetRandomNumber(0)), 0);

    EXPECT_LE(abs(GetRandomNumber(1)), 1);

    EXPECT_LE(abs(GetRandomNumber(1000000)), 1000000);

    EXPECT_LE(abs(GetRandomNumber(99999.999)), 99999.999);

    EXPECT_LE(abs(GetRandomNumber(0.0001)), 0.0001);

    EXPECT_LE(abs(GetRandomNumber(7)), 7);

    EXPECT_LE(abs(GetRandomNumber(123.4567)), 123.4567);
}


TEST(ThirdLabTests, InCircleCorrectResults) {
    EXPECT_EQ(InCircle(1, 0, 1), true);
    EXPECT_EQ(InCircle(0, 1, 1), true);
    EXPECT_EQ(InCircle(-1, 0, 1), true);
    EXPECT_EQ(InCircle(0, -1, 1), true);
    EXPECT_EQ(InCircle(1, 1, 1), false);
    EXPECT_EQ(InCircle(-1, -1, 1), false);
    EXPECT_EQ(InCircle(99999, -99999, 1), false);
    EXPECT_EQ(InCircle(-99999, 99999, 1), false);

    EXPECT_EQ(InCircle(0, 0, 0.1), true);
    EXPECT_EQ(InCircle(-0.000001, -0.000001, 0.1), true);
    EXPECT_EQ(InCircle(-0.1, -0.1, 0.000001), false);

    EXPECT_EQ(InCircle(3, 4, 5), true);
    EXPECT_EQ(InCircle(3.00001, 4.00000001, 5), false);

    EXPECT_EQ(InCircle(1234.5678, 9876.54321, 99999999), true);

}

vazy1@vazy1-legion:~/ClionProjects/OS-labs/tests$ ./../cmake-build-debug/tests/lab3_test 
Running main() from /home/botashev/ClionProjects/OS-labs/cmake-build-debug/_deps/googletest-src/googletest/src/gtest_main.cc
[==========] Running 2 tests from 1 test suite.
[----------] Global test environment set-up.
[----------] 2 tests from ThirdLabTests
[ RUN      ] ThirdLabTests.GetRandomNumberCorrectResults
[       OK ] ThirdLabTests.GetRandomNumberCorrectResults (0 ms)
[ RUN      ] ThirdLabTests.InCircleCorrectResults
[       OK ] ThirdLabTests.InCircleCorrectResults (0 ms)
[----------] 2 tests from ThirdLabTests (0 ms total)

[----------] Global test environment tear-down
[==========] 2 tests from 1 test suite ran. (0 ms total)
[  PASSED  ] 2 tests.

\end{verbatim}

\pagebreak

\section{Вывод}

Многие языки программирования позволяют пользователю работать с потоками. 
Создание потоков происходит быстрее, чем создание процессов, за счет того, что
при создании потока не копируется область памяти, а они все работают с одной
областью памяти. Поэтому многопоточность используют для ускарения не зависящих
друг от друга, однотипнях задач, которые будут работать параллельно.

Язык Си предоставляет данный функционал пользователям Unix-подобных операционных
систем с помощью библиотеки pthread.h. Средствами языка Си можно совершать системные
запросы на создание и ожидания завершения потока, а также использовать различные
примитивы синхронизации.
\end{document}
