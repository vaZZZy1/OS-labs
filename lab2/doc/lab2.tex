
\documentclass[pdf, unicode, 12pt, a4paper,oneside,fleqn]{article}
\usepackage{graphicx}
\graphicspath{{img/}}
\usepackage{log-style}
\begin{document}

\begin{titlepage}
    \begin{center}
        \bfseries

        {\Large Московский авиационный институт\\ (национальный исследовательский университет)}
        
        \vspace{48pt}
        
        {\large Факультет информационных технологий и прикладной математики}
        
        \vspace{36pt}
        
        {\large Кафедра вычислительной математики и~программирования}
        
        \vspace{48pt}
        
        Лабораторная работа \textnumero 2 по курсу \enquote{Операционные системы}

        \vspace{48pt}

        Управление процессами в ОС. Обеспечение обмена данных между процессами посредством каналов.
    \end{center}
    
    \vspace{140pt}
    
    \begin{flushright}
    \begin{tabular}{rl}
    Студент: & В.\,А. Амурский \\
    Преподаватель: & Е.\,С. Миронов \\
    Группа: & М8О-201Б-21 \\
    Вариант: & 11 \\
    Дата: & \\
    Оценка: & \\
    Подпись: & \\
    \end{tabular}
    \end{flushright}
    
    \vfill
    
    \begin{center}
    \bfseries
    Москва, \the\year
    \end{center}
\end{titlepage}
    
\pagebreak

\section{Постановка задачи}

Составить и отладить программу на языке Си, осуществляющую работу с процессами и
взаимодействие между ними в одной из двух операционных систем. В результате работы
программа (основной процесс) должен создать для решение задачи один или несколько
дочерних процессов. Взаимодействие между процессами осуществляется через системные
сигналы/события и/или каналы (pipe).

Необходимо обрабатывать системные ошибки, которые могут возникнуть в результате работы.

Родительский процесс создает два дочерних процесса. Перенаправление стандартных потоков
Child1 и Child2 можно «соединить» между собой
дополнительным каналом. Родительский и дочерний процесс должны быть представлены
разными программами.

Родительский процесс принимает от пользователя строки произвольной длины и пересылает их в
pipe1. Процесс child1 и child2 производят работу над строками. Child2 пересылает результат своей
работы родительскому процессу. Родительский процесс полученный результат выводит в
стандартный поток вывода.


Child1 переводит строки в верхний регистр. Child2 убирает все задвоенные пробелы.

\section{Сведения о программе}

Программа написанна на С++ в Unix подобной операционной системе на базе ядра Linux.

При запуске программы пользователь вводит строки в стандартный поток ввода. Программа создает два дочерних процесса для преобразования введенных строк.

По завершении работы программа выводит в стандартный поток вывода введенные строки в верхнем регистре, удалив все задвоенные пробелы

\section{Общий метод и алгоритм решения}

Родительский процесс создает первый дочерний процесс, передав через pipe1 строки, полученные от пользователя. Затем родительский процесс создает второй дочерний процесс.

Первый дочерний прочесс принимает строки и приводит все символы в верхний регистр, после чего передавая полученные строки во второй дочерний процесс через pipe3

Второй дочерний процесс принимает строки через pipe3, после чего удаляет все задвоенные пробелы и передает полученные строки родительскому процессу через pipe2

Результирующие строки родительский процесс считывает из pipe2

\section{Листинг программы}

{\large\textbf{main.cpp}}

\begin{lstlisting}[language=C++]
#include "include/parent.h"
#include <vector>

int main() {
    std::vector <std::string> input;

    std::string s;
    while (getline(std::cin, s)) {
        input.push_back(s);
    }

    std::vector <std::string> output = ParentRoutine("child1", "child2", input);

    for (const auto &res : output){
        std::cout << res << std::endl;
    }
    return 0;
}

\end{lstlisting}

{\large\textbf{parent.cpp}}

\begin{lstlisting}[language=C++]
#include "parent.h"


std::vector<std::string> ParentRoutine(char const *pathToChild1, char const *pathToChild2,
                                       const std::vector<std::string> &input){

    std::vector<std::string> output;

    int firstPipe[2];
    CreatePipe(firstPipe);
    int pipeBetweenChildren[2];
    CreatePipe(pipeBetweenChildren);

    int pid = fork();



    if (pid == 0) {

        close(firstPipe[WRITE_END]);
        close(pipeBetweenChildren[READ_END]);

        MakeDup2(firstPipe[READ_END], STDIN_FILENO);
        MakeDup2(pipeBetweenChildren[WRITE_END], STDOUT_FILENO);

        if (execl(pathToChild1, "", nullptr) == -1) {
            GetExecError(pathToChild1);
        }
        close(firstPipe[READ_END]);
        close(firstPipe[WRITE_END]);
    } else if (pid == -1) {
        GetForkError();
    } else {
        close(firstPipe[READ_END]);
        for (const std::string & s : input) {
            std::string s_tmp = s + "\n";
            write(firstPipe[WRITE_END], s_tmp.c_str(), s_tmp.size());
        }
        close(firstPipe[WRITE_END]);

        int secondPipe[2];
        CreatePipe(secondPipe);

        pid = fork();

        if (pid == 0) {
            close(secondPipe[READ_END]);
            close(pipeBetweenChildren[WRITE_END]);

            MakeDup2(pipeBetweenChildren[READ_END], STDIN_FILENO);
            MakeDup2(secondPipe[WRITE_END], STDOUT_FILENO);

            if (execl(pathToChild2, "", nullptr) == -1) {
                GetExecError(pathToChild2);
            }
            close(pipeBetweenChildren[READ_END]);
            close(secondPipe[WRITE_END]);
        } else if (pid == -1) {
            GetForkError();
        } else {
            close(secondPipe[WRITE_END]);
            close(pipeBetweenChildren[WRITE_END]);
            close(pipeBetweenChildren[READ_END]);

            wait(nullptr);

            for (size_t i = 0; i < input.size(); i++) {
                std::string res;
                char ch;
                while (read(secondPipe[READ_END], &ch, 1) && ch != '\n'){
                    res += ch;
                }
                output.push_back(res);
            }
            close(secondPipe[READ_END]);
        }
    }
    return output;
}
\end{lstlisting}

{\large\textbf{child1.cpp}}

\begin{lstlisting}[language=C++]
#include "utils.h"

int main() {
    std::string s;
    while(getline(std::cin, s)) {
        for (char & ch : s) {
            ch = toupper(ch);
        }
        std::cout << s << '\n';
    }
    return 0;
}
\end{lstlisting}

{\large\textbf{child2.cpp}}

\begin{lstlisting}[language=C++]
#include "utils.h"


int main() {
    std::string s;
    while (getline(std::cin, s)) {
        int j = 0;
        char lastCh = '\0';
        for (int i = 0; i < s.size(); i++){
            if (lastCh != ' ' || s[i] != ' '){
                s[j] = s[i];
                j++;
            }
            lastCh = s[i];
        }
        for (int i = 0; i < j; i++) {
            std::cout << s[i];
        }
        std::cout << '\n';
    }
    return 0;
}
\end{lstlisting}


{\large\textbf{utils.cpp}}

\begin{lstlisting}[language=C++]
#include "utils.h"

void CreatePipe(int fd[]) {
    if (pipe(fd) != 0) {
        std::cout << "Couldn't create pipe" << std::endl;
        exit(EXIT_FAILURE);
    }
}

void GetForkError() {
    std::cout << "fork error" << std::endl;
    exit(EXIT_FAILURE);
}

void MakeDup2(int oldFd, int newFd) {
    if (dup2(oldFd, newFd) == -1) {
        std::cout << "dup2 error" << std::endl;
        exit(EXIT_FAILURE);
    }
}

void GetExecError(std::string const &executableFile) {
    std::cout << "Exec \"" << executableFile << "\" error." << std::endl;
}

\end{lstlisting}

\section{Демонстрация работы программы}

\begin{verbatim}
vazy1@vazy1-legion:~/mai/OS-labs/tests$ cat lab2_test.cpp 
#include <gtest/gtest.h>

#include <array>
#include <memory>
#include <parent.h>
#include <stdlib.h>
#include <vector>


TEST(FirstLabTests, SimpleTest) {
	
	constexpr int inputSize = 4;
	
	std::array< std::vector<std::string>, inputSize> input;
	input[0] = {
		"abcabc",
		"ad SD da",
		"USE_LESS",
		"kEk sDf_doupf"
	};
	input[1] = {
		"  ",
		""
	};
	input[2] = {
		"__A  ",
		"_ _ _"
	};
	input[3] = {
		"\b "
	};
	std::array< std::vector<std::string>, inputSize> expectedOutput;
	expectedOutput[0] = {
		"ABCABC",
		"AD_SD_DA",
		"USE_LESS",
		"KEK_SDF_DOUPF"
	};
	expectedOutput[1] = {
		"__",
		""
	};
	expectedOutput[2] = {
		"__A__",
		"_____"
	};
	expectedOutput[3] = {
		"\b_"
	};
	for (int i = 0; i < inputSize; i++) {
		auto result = ParentProcces("/home/vazy1/mai/OS-labs/build/lab2/child1", "/home/vazy1/mai/OS-labs/build/lab2/child2", input[i]);
		EXPECT_EQ(result, expectedOutput[i]);
	}
}
vazy1@vazy1-legion:~/mai/OS-labs/build/tests$ /home/vazy1/mai/OS-labs/build/tests/lab2_test
Running main() from /home/vazy1/mai/OS-labs/build/_deps/googletest-src/googletest/src/gtest_main.cc
[==========] Running 1 test from 1 test suite.
[----------] Global test environment set-up.
[----------] 1 test from FirstLabTests
[ RUN      ] FirstLabTests.SimpleTest
[       OK ] FirstLabTests.SimpleTest (3 ms)
[----------] 1 test from FirstLabTests (3 ms total)

[----------] Global test environment tear-down
[==========] 1 test from 1 test suite ran. (4 ms total)
[  PASSED  ] 1 test.
\end{verbatim}

\pagebreak

\section{Вывод}

Одна из основных задач операционной системы - это управление процессами.
В большинстве случаев она сама создает процессы для себя и при запуске других программ.
Тем не менее бывают случаи, когда необходимо создавать процессы вручную.

В языке Си есть функционал, который позволит нам внутри нашей программы создать
дополнительный, дочерний процесс. Этот процесс будет работать параллельно с родительским.

Для этого в языке Си на Unix-подобных ОС используется библеотека unistd.h.
Эта библеотека позволяет совершать системные вызовы, которые связаны с 
вводом/выводом, управлением файлами, каталогами и работой с процессами и запуском программ.
Для создания дочерних процессов используется функция fork. При этом с помощью ветвлений 
в коде можно отделить код родителя от ребенка. У ребенка при этом можно заменить программу,
испрользуя для этого функцию exec, а обеспечить связь с помощью pipe.

Подобный функционал есть во многих языках программирования, так как большинство современных программ состаят более, чем из одного процесса.




\end{document}
