
\documentclass[pdf, unicode, 12pt, a4paper,oneside,fleqn]{article}
\usepackage{graphicx}
\usepackage{upquote}
\graphicspath{{img/}}
\usepackage{log-style}
\begin{document}

\begin{titlepage}
    \begin{center}
        \bfseries

        {\Large Московский авиационный институт\\ (национальный исследовательский университет)}
        
        \vspace{48pt}
        
        {\large Факультет информационных технологий и прикладной математики}
        
        \vspace{36pt}
        
        {\large Кафедра вычислительной математики и~программирования}
        
        \vspace{48pt}
        
        Лабораторная работа \textnumero 5 по курсу \enquote{Операционные системы}

        \vspace{48pt}

        Создание динамических библиотек. 
        
        Создание программ, которые используют функции динамических библиотек.
    \end{center}
    
    \vspace{125pt}
    
    \begin{flushright}
    \begin{tabular}{rl}
    Студент: & В.\,А. Амурский \\
    Преподаватель: & Е.\,С. Миронов \\
    Группа: & М8О-201Б-21 \\
    Вариант: & 9 \\
    Дата: & \\
    Оценка: & \\
    Подпись: & \\
    \end{tabular}
    \end{flushright}
    
    \vfill
    
    \begin{center}
    \bfseries
    Москва, \the\year
    \end{center}
\end{titlepage}
    
\pagebreak

\section{Постановка задачи}

Требуется создать динамические библиотеки, которые реализуют определенный функционал. 
Далее использовать данные библиотеки 2-мя способами:

\begin{itemize}
    \item Во время компиляции (на этапе «линковки»/linking)
    \item Во время исполнения программы. Библиотеки загружаются в память с помощью 
            интерфейса ОС для работы с динамическими библиотеками
\end{itemize}

В лабораторной работе необходимо получить следующие части:

\begin{itemize}
    \item Динамические библиотеки, реализующие контракты
    \item Тестовая программа (программа №1), которая используют одну из библиотек, используя 
    знания полученные на этапе компиляции.
    \item Тестовая программа (программа №2), которая загружает библиотеки, используя только их 
    местоположение и контракты.
\end{itemize}

Провести анализ двух типов использования библиотек.

Пользовательский ввод должен быть организован следующим образом:

\begin{itemize}
    \item Команда <<0>>: переключить одну реализацию контракты на другую
    \item Команда <<1 args>>: вызов первой функции контрактов
    \item Команда <<2 args>>: вызов второй функции контрактов
\end{itemize}

Контракты:
\begin{itemize}
    \item Подсчет количества простых чисел на промежутке [A; B]
    \item Отсортировать целочисленный массив
\end{itemize}

\section{Сведения о программе}

Программа написанна на Си в Unix подобной операционной системе на базе ядра Linux.

Оба контракта реализованы в двух вариантах:

Нахождение простых чисел на промежутке наивным способом(перебором делителей) и решетом Эратосфена

Сортировка массива пузырьком и quick-sort(сортировка Хоара)

Контракты описаны в файле signature.h, а реализация implem1.сpp и implem2.сpp

\begin{enumerate}
    \item Создание объектных файлов
    \item Компиляция библиотек с ключем -shared. Получаем динамические библиотеки с расширением .so 
    \item Линковка библиотеки к необходимой программе
\end{enumerate}

Для динамической загрузки библиотек используется библиотека dlfcn.h

\section{Общий метод и алгоритм решения}

Программа принимает в себя команды: 
\begin{itemize}
    \item В случае команды 1, мы считываем A и B и находим все простые числа на отрезке [A; B].
    \item В случае команды 2, мы считываем размер массива и сам массив и выводим его в отсортированном виде.
    \item В случае команды 0, мы закрываем старую библиотеку, открываем вторую и заменяем указатели на функции.
\end{itemize}

Для завершения программы нужно ввести комбинацию завершения ввода \--- CTRL+D.

\section{Листинг программы}

{\large\textbf{signature.h}}

\begin{lstlisting}[language=C++]
#ifndef OS-labs_SIGNATURE_H
#define OS-labs_SIGNATURE_H


#include <cstddef>
#include <algorithm>
#include <cmath>
#include <vector>

extern "C" {
    int primeCount(int a, int b);
    std::vector <int> sort(std::vector <int> array, int low, int high);
}


#endif //OS-labs_SIGNATURE_H

\end{lstlisting}

{\large\textbf{implem1.cpp}}

\begin{lstlisting}[language=C++]
#include "signature.h"

int primeCount(int a, int b) {
    int res = 0;
    for (int i = a; i <= b; i++){
        if (i > 1) {
            bool is_prime = true;
            for (int j = 2; j * j <= i; j++) {
                if (i % j == 0) {
                    is_prime = false;
                    break;
                }
            }
            if (is_prime) {
                res++;
            }
        }
    }
    return res;
}

std::vector <int>sort(std::vector <int>array, int low, int high) {
    for (int i = 0; i < high; i++) {
        for (int j = 0; j < high - 1; j++) {
            if (array[j] > array[j + 1]) {
                std::swap(array[j], array[j + 1]);
            }
        }
    }
    return array;
}

\end{lstlisting}

{\large\textbf{implem2.cpp}}

\begin{lstlisting}[language=C++]
#include "signature.h"
#include <stack>

int primeCount(int a, int b) {
    int prime[b + 1];
    for (int i = 0; i <= b; i++) {
        prime[i] = true;
    }
    prime[0] = prime[1] = false;
    int res = 0;
    for (int i = 2; i <= b; ++i)
        if (prime[i]) {
            if (i * 1ll * i <= b)
                for (int j = i * i; j <= b; j += i)
                    prime[j] = false;

            res += (i >= a);
        }
    return res;
}

std::vector <int> sort(std::vector <int>array, int low, int high) {
    int i = low;
    int j = high;
    int pivot = array[(i + j) / 2];

    while (i <= j)
    {
        while (array[i] < pivot)
            i++;
        while (array[j] > pivot)
            j--;
        if (i <= j)
        {
            std::swap(array[i], array[j]);
            i++;
            j--;
        }
    }
    if (j > low)
        array = sort(array, low, j);
    if (i < high)
        array = sort(array, i, high);

    return array;
}

\end{lstlisting}

{\large\textbf{main_static.cpp}}

\begin{lstlisting}[language=C++]
#include <iostream>
#include "signature.h"


int main() {
    int command;
    while (std::cin >> command) {
        if (command == 1) {
            int a, b;
            std::cin >> a >> b;
            if (a > b){
                std::cout << R"("a" have to be less or equal then "b"\n)";
            }
            else{
                std::cout << primeCount(a, b) << "\n";
            }

        } else if (command == 2) {
            int n;
            std::cin >> n;
            std::vector <int> arr(n);
            for (int i = 0; i < n; i++){
                std::cin >> arr[i];
            }
            arr = sort(arr, 0, n);

            for (int i = 0; i < n; i++){
                std::cout << arr[i] << " ";
            }
            std::cout << "\n";
        } else {
            std::cout << "you have to enter 1 or 2" << std::endl;
        }
    }
}

\end{lstlisting}

{\large\textbf{main_dynamic.cpp}}

\begin{lstlisting}[language=C++]
#include <array>
#include <cstdio>
#include <iostream>
#include <cstdlib>
#include <dlfcn.h>
#include <string>
#include <vector>


int main() {
    const std::vector <std::string> LIB = {"./libd1_dynamic.so", "./libd2_dynamic.so"};
    const std::vector <std::string> FUNC_NAME = {"primeCount", "sort"};
    int curlib = 0;
    int (*primeCount)(int a, int b);
    std::vector <int> (*sort)(std::vector <int>array, int low, int high);
    void* handle = dlopen(LIB[curlib].c_str(), RTLD_LAZY);
    if (handle == nullptr) {
        std::cout << "Fail dlopen\n";
        return EXIT_FAILURE;
    }
    primeCount = ((int (*)(int, int)) dlsym(handle, FUNC_NAME[0].c_str()));
    sort = (std::vector <int>(*)(std::vector <int>, int, int))dlsym(handle, FUNC_NAME[1].c_str());
    int command;
    while (std::cin >> command) {
        if (command == 1) {
            int a, b;
            std::cin >> a >> b;
            if (a > b){
                std::cout << R"("a" have to be less or equal then "b"\n)";
            }
            else{
                std::cout << primeCount(a, b) << "\n";
            }
        } else if (command == 2) {
            int n;
            std::cin >> n;
            std::vector <int>arr(n);
            for (int i = 0; i < n; i++){
                std::cin >> arr[i];
            }
            sort(arr, 0, n);

            for (int i = 0; i < n; i++){
                std::cout << arr[i] << " ";
            }
            std::cout << "\n";
        } else if (command == 0) {
            dlclose(handle);
            curlib ^= 1;
            void* handle = dlopen(LIB[curlib].c_str(), RTLD_LAZY);
            if (handle == nullptr) {
                std::cout << "Fail dlopen\n";
                return EXIT_FAILURE;
            }
            primeCount = ((int (*)(int, int)) dlsym(handle, FUNC_NAME[0].c_str()));
            sort = (std::vector <int>(*)(std::vector <int>, int, int))dlsym(handle, FUNC_NAME[1].c_str());
        } else {
            std::cout << "you have to enter 0, 1 or 2" << std::endl;
        }
    }
    dlclose(handle);
}

\end{lstlisting}

\pagebreak

\section{Демонстрация работы программы}


{\scriptsize
\begin{verbatim}

vazy1@vazy1-legion:~/ClionProjects/OS-labs/tests$ cat lab5_test.cpp
#include <fstream>
#include <gtest/gtest.h>
#include <string>
#include <dlfcn.h>
#include <signature.h>


TEST(Lab5Test, DynamicTest) {

    const std::vector <std::string> FUNC_NAME = {"primeCount", "sort"};

    std::vector < std::vector <int> > input1 = {{1, 10}, {1, 100}, {107, 107}, {100, 105}};

    std::vector < std::vector <int> > input2 = {
            {3, 2, 1, 4, 5},
            {100, -1, 0},
            {1, 1, 1, 1, 1},
            {1000000},
            {-1919, -66666, -789}
    };

    std::vector <int> output1 = {4, 25, 1, 2};

    std::vector < std::vector <int> > output2= {
            {1, 2, 3, 4, 5},
            {-1, 0, 100},
            {1, 1, 1, 1, 1},
            {1000000},
            {-66666, -1919, -789}
    };


    int (*primeCountOne)(int a, int b);
    std::vector <int> (*sortOne)(std::vector <int>array, int low, int high);
    void* handleOne = dlopen(
            getenv("PATH_TO_libd1_dynamic.so"), RTLD_LAZY);
    ASSERT_NE(handleOne, nullptr);


    int (*primeCountTwo)(int a, int b);
    std::vector <int>(*sortTwo)(std::vector <int>array, int low, int high);
    void* handleTwo = dlopen(getenv("PATH_TO_libd2_dynamic.so"), RTLD_LAZY);
    ASSERT_NE(handleTwo, nullptr);


    primeCountOne = ((int (*)(int, int)) dlsym(handleOne, FUNC_NAME[0].c_str()));
    primeCountTwo = ((int (*)(int, int)) dlsym(handleTwo, FUNC_NAME[0].c_str()));
    sortOne = (std::vector <int>(*)(std::vector <int>, int, int))dlsym(handleOne, FUNC_NAME[1].c_str());
    sortTwo = (std::vector <int>(*)(std::vector <int>, int, int))dlsym(handleTwo, FUNC_NAME[1].c_str());


    for (size_t i = 0; i < input1.size(); i++) {
        auto primeCountOutOne = primeCountOne(input1[i][0], input1[i][1]);
        auto primeCountOutTwo = primeCountTwo(input1[i][0], input1[i][1]);
        EXPECT_EQ(primeCountOutOne, output1[i]);
        EXPECT_EQ(primeCountOutTwo, output1[i]);
    }

    for (size_t i = 0; i < input2.size(); i++) {
        auto sortOutOne = sortOne(input2[i], 0, (int)input2[i].size());
        auto sortOutTwo = sortTwo(input2[i], 0, (int)input2[i].size() - 1);


        EXPECT_EQ(sortOutOne, output2[i]);
        EXPECT_EQ(sortOutTwo, output2[i]);
    }
}

TEST(Lab5Test, StaticOneTest) {

    std::vector < std::vector <int> > input1 = {{1, 10}, {1, 100}, {107, 107}, {100, 105}};

    std::vector < std::vector <int> > input2 = {
            {3, 2, 1, 4, 5},
            {100, -1, 0},
            {1, 1, 1, 1, 1},
            {1000000},
            {-1919, -66666, -789}
    };

    std::vector <int> output1 = {4, 25, 1, 2};

    std::vector < std::vector <int> > output2= {
            {1, 2, 3, 4, 5},
            {-1, 0, 100},
            {1, 1, 1, 1, 1},
            {1000000},
            {-66666, -1919, -789}
    };

    for (int i = 0; i < input1.size(); i++) {
        auto primeCountOut = primeCount(input1[i][0], input1[i][1]);
        EXPECT_EQ(primeCountOut, output1[i]);
    }

    for (size_t i = 0; i < input2.size(); i++) {
        auto sortOut = sort(input2[i], 0, (int)input2[i].size());

        for (int j = 0; j < output2[i].size(); j++){
            EXPECT_EQ(sortOut[j], output2[i][j]);
        }
    }
}
vazy1@vazy1-legion:~/ClionProjects/OS-labs/tests$ ./../cmake-build-debug/tests/lab5_test 
Running main() from /home/botashev/ClionProjects/OS-labs/cmake-build-debug/_deps/googletest-src/googletest/src/gtest_main.cc
[==========] Running 2 tests from 1 test suite.
[----------] Global test environment set-up.
[----------] 2 tests from Lab5Test
[ RUN      ] Lab5Test.DynamicTest
[       OK ] Lab5Test.DynamicTest (1 ms)
[ RUN      ] Lab5Test.StaticOneTest
[       OK ] Lab5Test.StaticOneTest (0 ms)
[----------] 2 tests from Lab5Test (1 ms total)

[----------] Global test environment tear-down
[==========] 2 tests from 1 test suite ran. (1 ms total)
[  PASSED  ] 2 tests.
\end{verbatim}
}

\pagebreak

\section{Вывод}

Лабораторная работа была направлена на изучение динамических библиотек в Unix подобных операционных
системах. Для изучения создания и работы с ними мною было написанно 2 программ: одна
подлючала динамичкие библиотеки на этапе компиляции, а вторая во время исполнения.

Динамические библиотеки содержат функционал отдельно от программы и прередают его
непосредственно во время исполнения. Из плюсов такого подхода можно выделить, что 
во-первых, в таком случае размер результирующей программы меньше,во-вторых, одну и ту же библиотеку 
можно использовать в нескольких программах не встраивая в код, чем можно также добиться снижения
общего занимаемого пространства на диске, и в-третьих, что после исправления ошибок
в библиотеке не нужно перекомпилировать все программы, достаточно перекомпилировать саму библиотеку.

Однако у динамических библиотек есть и недостатки. Первый заключается в том, что
вызов функции из динамической библиотеки происходит медленнее. Второй, что
мы не можем подправить функционал библиотеки под конкретную программу не зацепив
при этом других программ, работающих с этой библиотекой. И в-третьих, уже скомпилированная
программа не будет работать на аналогичной системе без установленной динамической библиотеки.

Тем не менее плюсы динамичеких библиотек исчерпывают их минусы в большенстве задач, 
обратных случаях лучше обратиться к статическим библиотекам. В наше время с высокими мощностями 
вычислительных систем становится более важным сэкономить объем памяти, используемый программой,
чем время обращения к функции. Поэтому динамические библиотеки используются в большенстве 
современных программ.
\end{document}
